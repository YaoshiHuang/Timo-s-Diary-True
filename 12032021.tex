\documentclass[11pt]{ctexart}
\usepackage[margin=2cm,a4paper]{geometry}

\setmainfont{Caladea}

%% 也可以选用其它字库:
% \setCJKmainfont[%
%   ItalicFont=AR PL KaitiM GB,
%   BoldFont=Noto Sans CJK SC,
% ]{Noto Serif CJK SC}
% \setCJKsansfont{Noto Sans CJK SC}
% \renewcommand{\kaishu}{\CJKfontspec{AR PL KaitiM GB}}


\usepackage{minted}

\usepackage[breaklinks]{hyperref}
\title{Diary of depression Day 11}
\author{Timo}

\begin{document}
\maketitle

焦虑症的第十一天,一切仿佛看到了希望。\par
之前由于痛苦太大,日记中断了一个多星期。在极强的副作用之下,头痛欲裂,却依然找不到任何希望,沉重和麻木蔓延全身,窝在床上抽搐,却连哭喊的能力都丧失了。我想要逃离,却不知道逃向何方,甚至想过逃避世俗的定义,去当个流浪汉或者街头艺人。\par
奇诺之旅里有一个国家,那里的人们从生到死,只为了建造一座通天塔,当通天塔不支倒下的时候,人们则庆祝下一座的奠基,周而复始。有一个男孩子见到奇诺,忍不住问了,建造通天塔的意义,并央求奇诺带他出去看看。而奇诺问他,你不建造通天塔,又想做什么呢?男孩支支吾吾答不上来,他只是想要离开,最终奇诺并没有带他离开。一个想要离开世俗的桎梏的人,却并不知道离开世俗之后,还有什么事可做,这样的人,别人是帮不了的。\par
还好我始终没有放弃和逃避,家人,医生,老师,朋友们,都没有放弃。彻底失去行为能力的一个周,反而让我静下来,停止抱怨,开始反思,一点一点拼凑起自我。家庭对我的影响,从哪里来,又投射到哪里;我和父亲、母亲、其他人的关系,是否有边界,又是否是多元和稳固的;抛开人即工具的思维模式,我所追求的又是什么。这些问题,虽然有点迟,但是总算是开始有了一些答案。\par
生活,仿佛也开始恢复秩序了。

\end{document}