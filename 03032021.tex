\documentclass[11pt]{ctexart}
\usepackage[margin=2cm,a4paper]{geometry}

\setmainfont{Caladea}

%% 也可以选用其它字库:
% \setCJKmainfont[%
%   ItalicFont=AR PL KaitiM GB,
%   BoldFont=Noto Sans CJK SC,
% ]{Noto Serif CJK SC}
% \setCJKsansfont{Noto Sans CJK SC}
% \renewcommand{\kaishu}{\CJKfontspec{AR PL KaitiM GB}}


\usepackage{minted}

\usepackage[breaklinks]{hyperref}
\title{Diary of depression Day 2}
\author{Timo}

\begin{document}
\maketitle

记录焦虑的第二天,可能只有写点东西,能让我短暂地平静下来。并没有特别后悔,当时选择的专业和方向,但是也不得不做出,要及时止损的艰难决定。与其抱怨,方向如何混乱,决策如何失当,不如静下来问自己,为什么会这样,以后该怎么办。\par
无论你是学士,硕士,博士,其实登峰造极和碌碌无为,只有一线的区别。找到自己的路,坚定地走下去,加上一些运气,登峰造极是自然而然的;找错了路,来回切换,犹豫不决,运气自然也不会找上门来,就注定碌碌无为。\par
我从小可能就是个当老师的材料,一站上讲台,就刹不住车。因为家里全都是光荣的人民教师,我自然不会干别的。让我去做买卖,去从政,我是不太会的,天生没有那么多的心眼;搞音乐艺术,我又偏偏是爱听老歌,爱唱京剧,跟不上节奏;搞科研嘛,曾经很热爱,听上去也很有趣,可以一试。于是,怀着当高校老师梦想的我,毅然地走上了读博的道路。\par
时过境迁,阴差阳错,为理想而奋斗、为主义而牺牲的豪言壮语,湮灭在现实的残酷中,我也没有想到,研究方向走到了今天,走入一片无人问津的死海,对科研的热爱,再也无从谈起。我不知道这是命运的玩笑,还是家族轮回的必然,也不知道时至今日蓦然回首,是不是为时已晚。从我个人的角度,我只能说,价值观的坍塌与重构,由内而外的颠覆。浴火,未必可以重生,希望我能坚持到日出的时候。\par


\end{document}